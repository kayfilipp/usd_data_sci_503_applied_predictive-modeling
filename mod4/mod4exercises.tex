% Options for packages loaded elsewhere
\PassOptionsToPackage{unicode}{hyperref}
\PassOptionsToPackage{hyphens}{url}
%
\documentclass[
]{article}
\usepackage{amsmath,amssymb}
\usepackage{lmodern}
\usepackage{ifxetex,ifluatex}
\ifnum 0\ifxetex 1\fi\ifluatex 1\fi=0 % if pdftex
  \usepackage[T1]{fontenc}
  \usepackage[utf8]{inputenc}
  \usepackage{textcomp} % provide euro and other symbols
\else % if luatex or xetex
  \usepackage{unicode-math}
  \defaultfontfeatures{Scale=MatchLowercase}
  \defaultfontfeatures[\rmfamily]{Ligatures=TeX,Scale=1}
\fi
% Use upquote if available, for straight quotes in verbatim environments
\IfFileExists{upquote.sty}{\usepackage{upquote}}{}
\IfFileExists{microtype.sty}{% use microtype if available
  \usepackage[]{microtype}
  \UseMicrotypeSet[protrusion]{basicmath} % disable protrusion for tt fonts
}{}
\makeatletter
\@ifundefined{KOMAClassName}{% if non-KOMA class
  \IfFileExists{parskip.sty}{%
    \usepackage{parskip}
  }{% else
    \setlength{\parindent}{0pt}
    \setlength{\parskip}{6pt plus 2pt minus 1pt}}
}{% if KOMA class
  \KOMAoptions{parskip=half}}
\makeatother
\usepackage{xcolor}
\IfFileExists{xurl.sty}{\usepackage{xurl}}{} % add URL line breaks if available
\IfFileExists{bookmark.sty}{\usepackage{bookmark}}{\usepackage{hyperref}}
\hypersetup{
  pdftitle={Module 4 exercises},
  pdfauthor={Filipp Krasovsky},
  hidelinks,
  pdfcreator={LaTeX via pandoc}}
\urlstyle{same} % disable monospaced font for URLs
\usepackage[margin=1in]{geometry}
\usepackage{color}
\usepackage{fancyvrb}
\newcommand{\VerbBar}{|}
\newcommand{\VERB}{\Verb[commandchars=\\\{\}]}
\DefineVerbatimEnvironment{Highlighting}{Verbatim}{commandchars=\\\{\}}
% Add ',fontsize=\small' for more characters per line
\usepackage{framed}
\definecolor{shadecolor}{RGB}{248,248,248}
\newenvironment{Shaded}{\begin{snugshade}}{\end{snugshade}}
\newcommand{\AlertTok}[1]{\textcolor[rgb]{0.94,0.16,0.16}{#1}}
\newcommand{\AnnotationTok}[1]{\textcolor[rgb]{0.56,0.35,0.01}{\textbf{\textit{#1}}}}
\newcommand{\AttributeTok}[1]{\textcolor[rgb]{0.77,0.63,0.00}{#1}}
\newcommand{\BaseNTok}[1]{\textcolor[rgb]{0.00,0.00,0.81}{#1}}
\newcommand{\BuiltInTok}[1]{#1}
\newcommand{\CharTok}[1]{\textcolor[rgb]{0.31,0.60,0.02}{#1}}
\newcommand{\CommentTok}[1]{\textcolor[rgb]{0.56,0.35,0.01}{\textit{#1}}}
\newcommand{\CommentVarTok}[1]{\textcolor[rgb]{0.56,0.35,0.01}{\textbf{\textit{#1}}}}
\newcommand{\ConstantTok}[1]{\textcolor[rgb]{0.00,0.00,0.00}{#1}}
\newcommand{\ControlFlowTok}[1]{\textcolor[rgb]{0.13,0.29,0.53}{\textbf{#1}}}
\newcommand{\DataTypeTok}[1]{\textcolor[rgb]{0.13,0.29,0.53}{#1}}
\newcommand{\DecValTok}[1]{\textcolor[rgb]{0.00,0.00,0.81}{#1}}
\newcommand{\DocumentationTok}[1]{\textcolor[rgb]{0.56,0.35,0.01}{\textbf{\textit{#1}}}}
\newcommand{\ErrorTok}[1]{\textcolor[rgb]{0.64,0.00,0.00}{\textbf{#1}}}
\newcommand{\ExtensionTok}[1]{#1}
\newcommand{\FloatTok}[1]{\textcolor[rgb]{0.00,0.00,0.81}{#1}}
\newcommand{\FunctionTok}[1]{\textcolor[rgb]{0.00,0.00,0.00}{#1}}
\newcommand{\ImportTok}[1]{#1}
\newcommand{\InformationTok}[1]{\textcolor[rgb]{0.56,0.35,0.01}{\textbf{\textit{#1}}}}
\newcommand{\KeywordTok}[1]{\textcolor[rgb]{0.13,0.29,0.53}{\textbf{#1}}}
\newcommand{\NormalTok}[1]{#1}
\newcommand{\OperatorTok}[1]{\textcolor[rgb]{0.81,0.36,0.00}{\textbf{#1}}}
\newcommand{\OtherTok}[1]{\textcolor[rgb]{0.56,0.35,0.01}{#1}}
\newcommand{\PreprocessorTok}[1]{\textcolor[rgb]{0.56,0.35,0.01}{\textit{#1}}}
\newcommand{\RegionMarkerTok}[1]{#1}
\newcommand{\SpecialCharTok}[1]{\textcolor[rgb]{0.00,0.00,0.00}{#1}}
\newcommand{\SpecialStringTok}[1]{\textcolor[rgb]{0.31,0.60,0.02}{#1}}
\newcommand{\StringTok}[1]{\textcolor[rgb]{0.31,0.60,0.02}{#1}}
\newcommand{\VariableTok}[1]{\textcolor[rgb]{0.00,0.00,0.00}{#1}}
\newcommand{\VerbatimStringTok}[1]{\textcolor[rgb]{0.31,0.60,0.02}{#1}}
\newcommand{\WarningTok}[1]{\textcolor[rgb]{0.56,0.35,0.01}{\textbf{\textit{#1}}}}
\usepackage{graphicx}
\makeatletter
\def\maxwidth{\ifdim\Gin@nat@width>\linewidth\linewidth\else\Gin@nat@width\fi}
\def\maxheight{\ifdim\Gin@nat@height>\textheight\textheight\else\Gin@nat@height\fi}
\makeatother
% Scale images if necessary, so that they will not overflow the page
% margins by default, and it is still possible to overwrite the defaults
% using explicit options in \includegraphics[width, height, ...]{}
\setkeys{Gin}{width=\maxwidth,height=\maxheight,keepaspectratio}
% Set default figure placement to htbp
\makeatletter
\def\fps@figure{htbp}
\makeatother
\setlength{\emergencystretch}{3em} % prevent overfull lines
\providecommand{\tightlist}{%
  \setlength{\itemsep}{0pt}\setlength{\parskip}{0pt}}
\setcounter{secnumdepth}{-\maxdimen} % remove section numbering
\ifluatex
  \usepackage{selnolig}  % disable illegal ligatures
\fi

\title{Module 4 exercises}
\author{Filipp Krasovsky}
\date{6/7/2021}

\begin{document}
\maketitle

\begin{enumerate}
\def\labelenumi{\arabic{enumi}.}
\tightlist
\item
  (15 points) Simulate a single predictor and a nonlinear relationship,
  such as a sin wave shown in Fig. 7.7 of textbook, and investigate the
  relationship between the cost, 𝜺, and kernel parameters for a support
  vector machine model:
\end{enumerate}

\hypertarget{create-a-grid-of-x-values-to-use-for-prediction-datagrid---data.framex-seq2-10-length-100.}{%
\subsection{Create a grid of x values to use for prediction
\textgreater{} dataGrid \textless- data.frame(x = seq(2, 10, length =
100)).}\label{create-a-grid-of-x-values-to-use-for-prediction-datagrid---data.framex-seq2-10-length-100.}}

\begin{Shaded}
\begin{Highlighting}[]
\CommentTok{\#training data}
\FunctionTok{set.seed}\NormalTok{(}\DecValTok{100}\NormalTok{)}
\NormalTok{x}\OtherTok{\textless{}{-}}\FunctionTok{runif}\NormalTok{(}\DecValTok{100}\NormalTok{,}\AttributeTok{min=}\DecValTok{2}\NormalTok{,}\AttributeTok{max=}\DecValTok{10}\NormalTok{)}
\NormalTok{y}\OtherTok{\textless{}{-}}\FunctionTok{sin}\NormalTok{(x)}\SpecialCharTok{+}\FunctionTok{rnorm}\NormalTok{(}\FunctionTok{length}\NormalTok{(x)) }\SpecialCharTok{*} \FloatTok{0.25}
\NormalTok{sinData}\OtherTok{\textless{}{-}}\FunctionTok{data.frame}\NormalTok{(}\AttributeTok{x=}\NormalTok{x,}\AttributeTok{y=}\NormalTok{y)}
\FunctionTok{plot}\NormalTok{(x,y,}\AttributeTok{main =} \StringTok{"Training values"}\NormalTok{)}
\end{Highlighting}
\end{Shaded}

\includegraphics{mod4exercises_files/figure-latex/unnamed-chunk-1-1.pdf}

\begin{Shaded}
\begin{Highlighting}[]
\CommentTok{\#prediction data }
\NormalTok{x\_test }\OtherTok{\textless{}{-}} \FunctionTok{data.frame}\NormalTok{(}\AttributeTok{x=}\FunctionTok{seq}\NormalTok{(}\DecValTok{2}\NormalTok{,}\DecValTok{10}\NormalTok{,}\AttributeTok{length=}\DecValTok{100}\NormalTok{))}
\NormalTok{y\_test }\OtherTok{\textless{}{-}} \FunctionTok{sin}\NormalTok{(x\_test)}\SpecialCharTok{+}\FunctionTok{rnorm}\NormalTok{(}\FunctionTok{length}\NormalTok{(x\_test))}\SpecialCharTok{*} \FloatTok{0.25}
\end{Highlighting}
\end{Shaded}

\begin{enumerate}
\def\labelenumi{(\alph{enumi})}
\tightlist
\item
  Fit different models using a radial basis function and different
  values of the cost (the C parameter) and 𝜺. Plot the fitted curve. For
  example: library(kernlab) rbfSVM \textless- ksvm(x = x, y = y, data =
  sinData, kernel =``rbfdot'', kpar = ``automatic'', C = 1, epsilon =
  0.1) modelPrediction \textless- predict(rbfSVM, newdata = dataGrid)
\end{enumerate}

We start with a cost param of 1 and an epsilon of 0.1.

\begin{Shaded}
\begin{Highlighting}[]
\CommentTok{\#use the ksvm model for svm with rbfdot as our kernel method.}
\FunctionTok{require}\NormalTok{(kernlab)}
\end{Highlighting}
\end{Shaded}

\begin{verbatim}
## Loading required package: kernlab
\end{verbatim}

\begin{verbatim}
## 
## Attaching package: 'kernlab'
\end{verbatim}

\begin{verbatim}
## The following object is masked from 'package:ggplot2':
## 
##     alpha
\end{verbatim}

\begin{Shaded}
\begin{Highlighting}[]
\FunctionTok{set.seed}\NormalTok{(}\DecValTok{100}\NormalTok{)}
\NormalTok{svm }\OtherTok{\textless{}{-}} \FunctionTok{ksvm}\NormalTok{(}\AttributeTok{x=}\NormalTok{x,}\AttributeTok{y=}\NormalTok{y,}\AttributeTok{data=}\NormalTok{sinData,}\AttributeTok{kernel=}\StringTok{"rbfdot"}\NormalTok{,}\AttributeTok{kpar=}\StringTok{"automatic"}\NormalTok{,}\AttributeTok{C=}\DecValTok{1}\NormalTok{,}\AttributeTok{epsilon=}\FloatTok{0.1}\NormalTok{)}
\NormalTok{svm.p}\OtherTok{\textless{}{-}}\FunctionTok{predict}\NormalTok{(svm,}\AttributeTok{newdata=}\NormalTok{x\_test)}
\FunctionTok{plot}\NormalTok{(}\AttributeTok{x =}\NormalTok{ svm.p, }\AttributeTok{y =}\NormalTok{ y\_test}\SpecialCharTok{$}\NormalTok{x,}\AttributeTok{main=}\FunctionTok{paste}\NormalTok{(}\StringTok{"C: 1, Eps = 0.1, RMSE:"}\NormalTok{,}\FunctionTok{round}\NormalTok{(}\FunctionTok{RMSE}\NormalTok{(svm.p,y\_test}\SpecialCharTok{$}\NormalTok{x),}\DecValTok{3}\NormalTok{)))}
\end{Highlighting}
\end{Shaded}

\includegraphics{mod4exercises_files/figure-latex/unnamed-chunk-2-1.pdf}

We move on to an epsilon of 0.075 and a cost param of 1:

\begin{Shaded}
\begin{Highlighting}[]
\CommentTok{\#use the ksvm model for svm with rbfdot as our kernel method.}
\FunctionTok{set.seed}\NormalTok{(}\DecValTok{100}\NormalTok{)}
\NormalTok{svm }\OtherTok{\textless{}{-}} \FunctionTok{ksvm}\NormalTok{(}\AttributeTok{x=}\NormalTok{x,}\AttributeTok{y=}\NormalTok{y,}\AttributeTok{data=}\NormalTok{sinData,}\AttributeTok{kernel=}\StringTok{"rbfdot"}\NormalTok{,}\AttributeTok{kpar=}\StringTok{"automatic"}\NormalTok{,}\AttributeTok{C=}\DecValTok{1}\NormalTok{,}\AttributeTok{epsilon=}\FloatTok{0.075}\NormalTok{)}
\NormalTok{svm.p}\OtherTok{\textless{}{-}}\FunctionTok{predict}\NormalTok{(svm,}\AttributeTok{newdata=}\NormalTok{x\_test)}
\FunctionTok{plot}\NormalTok{(}\AttributeTok{x =}\NormalTok{ svm.p, }\AttributeTok{y =}\NormalTok{ y\_test}\SpecialCharTok{$}\NormalTok{x,}\AttributeTok{main=}\FunctionTok{paste}\NormalTok{(}\StringTok{"C: 1, Eps = 0.075, RMSE:"}\NormalTok{,}\FunctionTok{round}\NormalTok{(}\FunctionTok{RMSE}\NormalTok{(svm.p,y\_test}\SpecialCharTok{$}\NormalTok{x),}\DecValTok{3}\NormalTok{)))}
\end{Highlighting}
\end{Shaded}

\includegraphics{mod4exercises_files/figure-latex/unnamed-chunk-3-1.pdf}
It appears decreasing out epsilon slightly drops our performance. We
move on to an epsilon of 0.125 and a cost param of 1:

\begin{Shaded}
\begin{Highlighting}[]
\CommentTok{\#use the ksvm model for svm with rbfdot as our kernel method.}
\FunctionTok{set.seed}\NormalTok{(}\DecValTok{100}\NormalTok{)}
\NormalTok{svm }\OtherTok{\textless{}{-}} \FunctionTok{ksvm}\NormalTok{(}\AttributeTok{x=}\NormalTok{x,}\AttributeTok{y=}\NormalTok{y,}\AttributeTok{data=}\NormalTok{sinData,}\AttributeTok{kernel=}\StringTok{"rbfdot"}\NormalTok{,}\AttributeTok{kpar=}\StringTok{"automatic"}\NormalTok{,}\AttributeTok{C=}\DecValTok{1}\NormalTok{,}\AttributeTok{epsilon=}\FloatTok{0.125}\NormalTok{)}
\NormalTok{svm.p}\OtherTok{\textless{}{-}}\FunctionTok{predict}\NormalTok{(svm,}\AttributeTok{newdata=}\NormalTok{x\_test)}
\FunctionTok{plot}\NormalTok{(}\AttributeTok{x =}\NormalTok{ svm.p, }\AttributeTok{y =}\NormalTok{ y\_test}\SpecialCharTok{$}\NormalTok{x,}\AttributeTok{main=}\FunctionTok{paste}\NormalTok{(}\StringTok{"C: 1, Eps = 0.125, RMSE:"}\NormalTok{,}\FunctionTok{round}\NormalTok{(}\FunctionTok{RMSE}\NormalTok{(svm.p,y\_test}\SpecialCharTok{$}\NormalTok{x),}\DecValTok{3}\NormalTok{)))}
\end{Highlighting}
\end{Shaded}

\includegraphics{mod4exercises_files/figure-latex/unnamed-chunk-4-1.pdf}
For boundary testing the cost parameter, we can scale C across 0.75 and
1.25 as an exercise, hosting epsilon constant at 0.1. As an extreme
boundary, we can set cost to near zero and five as well.

\begin{Shaded}
\begin{Highlighting}[]
\CommentTok{\#use the ksvm model for svm with rbfdot as our kernel method.}
\FunctionTok{par}\NormalTok{(}\AttributeTok{mfrow=}\FunctionTok{c}\NormalTok{(}\DecValTok{2}\NormalTok{,}\DecValTok{2}\NormalTok{))}
\FunctionTok{set.seed}\NormalTok{(}\DecValTok{100}\NormalTok{)}
\NormalTok{svm }\OtherTok{\textless{}{-}} \FunctionTok{ksvm}\NormalTok{(}\AttributeTok{x=}\NormalTok{x,}\AttributeTok{y=}\NormalTok{y,}\AttributeTok{data=}\NormalTok{sinData,}\AttributeTok{kernel=}\StringTok{"rbfdot"}\NormalTok{,}\AttributeTok{kpar=}\StringTok{"automatic"}\NormalTok{,}\AttributeTok{C=}\FloatTok{0.75}\NormalTok{,}\AttributeTok{epsilon=}\FloatTok{0.1}\NormalTok{)}
\NormalTok{svm.p}\OtherTok{\textless{}{-}}\FunctionTok{predict}\NormalTok{(svm,}\AttributeTok{newdata=}\NormalTok{x\_test)}
\FunctionTok{plot}\NormalTok{(}\AttributeTok{x =}\NormalTok{ svm.p, }\AttributeTok{y =}\NormalTok{ y\_test}\SpecialCharTok{$}\NormalTok{x,}\AttributeTok{main=}\FunctionTok{paste}\NormalTok{(}\StringTok{"C: 0.75, Eps = 0.1, RMSE:"}\NormalTok{,}\FunctionTok{round}\NormalTok{(}\FunctionTok{RMSE}\NormalTok{(svm.p,y\_test}\SpecialCharTok{$}\NormalTok{x),}\DecValTok{3}\NormalTok{)))}

\FunctionTok{set.seed}\NormalTok{(}\DecValTok{100}\NormalTok{)}
\NormalTok{svm }\OtherTok{\textless{}{-}} \FunctionTok{ksvm}\NormalTok{(}\AttributeTok{x=}\NormalTok{x,}\AttributeTok{y=}\NormalTok{y,}\AttributeTok{data=}\NormalTok{sinData,}\AttributeTok{kernel=}\StringTok{"rbfdot"}\NormalTok{,}\AttributeTok{kpar=}\StringTok{"automatic"}\NormalTok{,}\AttributeTok{C=}\FloatTok{1.25}\NormalTok{,}\AttributeTok{epsilon=}\FloatTok{0.1}\NormalTok{)}
\NormalTok{svm.p}\OtherTok{\textless{}{-}}\FunctionTok{predict}\NormalTok{(svm,}\AttributeTok{newdata=}\NormalTok{x\_test)}
\FunctionTok{plot}\NormalTok{(}\AttributeTok{x =}\NormalTok{ svm.p, }\AttributeTok{y =}\NormalTok{ y\_test}\SpecialCharTok{$}\NormalTok{x,}\AttributeTok{main=}\FunctionTok{paste}\NormalTok{(}\StringTok{"C: 1.25, Eps = 0.1, RMSE:"}\NormalTok{,}\FunctionTok{round}\NormalTok{(}\FunctionTok{RMSE}\NormalTok{(svm.p,y\_test}\SpecialCharTok{$}\NormalTok{x),}\DecValTok{3}\NormalTok{)))}

\FunctionTok{set.seed}\NormalTok{(}\DecValTok{100}\NormalTok{)}
\NormalTok{svm }\OtherTok{\textless{}{-}} \FunctionTok{ksvm}\NormalTok{(}\AttributeTok{x=}\NormalTok{x,}\AttributeTok{y=}\NormalTok{y,}\AttributeTok{data=}\NormalTok{sinData,}\AttributeTok{kernel=}\StringTok{"rbfdot"}\NormalTok{,}\AttributeTok{kpar=}\StringTok{"automatic"}\NormalTok{,}\AttributeTok{C=}\FloatTok{0.001}\NormalTok{,}\AttributeTok{epsilon=}\FloatTok{0.1}\NormalTok{)}
\NormalTok{svm.p}\OtherTok{\textless{}{-}}\FunctionTok{predict}\NormalTok{(svm,}\AttributeTok{newdata=}\NormalTok{x\_test)}
\FunctionTok{plot}\NormalTok{(}\AttributeTok{x =}\NormalTok{ svm.p, }\AttributeTok{y =}\NormalTok{ y\_test}\SpecialCharTok{$}\NormalTok{x,}\AttributeTok{main=}\FunctionTok{paste}\NormalTok{(}\StringTok{"C: \textasciitilde{}0, Eps = 0.1, RMSE:"}\NormalTok{,}\FunctionTok{round}\NormalTok{(}\FunctionTok{RMSE}\NormalTok{(svm.p,y\_test}\SpecialCharTok{$}\NormalTok{x),}\DecValTok{3}\NormalTok{)))}

\FunctionTok{set.seed}\NormalTok{(}\DecValTok{100}\NormalTok{)}
\NormalTok{svm }\OtherTok{\textless{}{-}} \FunctionTok{ksvm}\NormalTok{(}\AttributeTok{x=}\NormalTok{x,}\AttributeTok{y=}\NormalTok{y,}\AttributeTok{data=}\NormalTok{sinData,}\AttributeTok{kernel=}\StringTok{"rbfdot"}\NormalTok{,}\AttributeTok{kpar=}\StringTok{"automatic"}\NormalTok{,}\AttributeTok{C=}\DecValTok{5}\NormalTok{,}\AttributeTok{epsilon=}\FloatTok{0.1}\NormalTok{)}
\NormalTok{svm.p}\OtherTok{\textless{}{-}}\FunctionTok{predict}\NormalTok{(svm,}\AttributeTok{newdata=}\NormalTok{x\_test)}
\FunctionTok{plot}\NormalTok{(}\AttributeTok{x =}\NormalTok{ svm.p, }\AttributeTok{y =}\NormalTok{ y\_test}\SpecialCharTok{$}\NormalTok{x,}\AttributeTok{main=}\FunctionTok{paste}\NormalTok{(}\StringTok{"C: 5, Eps = 0.1, RMSE:"}\NormalTok{,}\FunctionTok{round}\NormalTok{(}\FunctionTok{RMSE}\NormalTok{(svm.p,y\_test}\SpecialCharTok{$}\NormalTok{x),}\DecValTok{3}\NormalTok{)))}
\end{Highlighting}
\end{Shaded}

\includegraphics{mod4exercises_files/figure-latex/unnamed-chunk-5-1.pdf}
On face value, it appears that increasing the cost parameter is
associated with an increase in RMSE, with diminishing marginal returns.
Near zero cost parameters induce a much larger RMSE, while higher
C-values decrease performance as well.

\begin{enumerate}
\def\labelenumi{(\alph{enumi})}
\setcounter{enumi}{1}
\tightlist
\item
  The σ parameter can be adjusted using the kpar argument, such as kpar
  = list(sigma = 1). Try different values of σ to understand how this
  parameter changes the model fit. How do the cost, 𝜺, and σ values
  affect the model?
\end{enumerate}

We know, as a rule, that higher values of sigma make the model more
flexible because errors tend to get amplified, but lower values make the
model ``stiffen'', leading to underfitting.

\begin{Shaded}
\begin{Highlighting}[]
\NormalTok{sigma\_frame }\OtherTok{=} \FunctionTok{seq}\NormalTok{(}\AttributeTok{from =} \FloatTok{0.1}\NormalTok{, }\AttributeTok{to =} \DecValTok{4}\NormalTok{, }\AttributeTok{by =} \FloatTok{0.1}\NormalTok{)}
\NormalTok{results }\OtherTok{=} \FunctionTok{data.frame}\NormalTok{(}\AttributeTok{Sigma=}\FunctionTok{character}\NormalTok{(),}\AttributeTok{RMSE=}\FunctionTok{double}\NormalTok{())}

\ControlFlowTok{for}\NormalTok{ (i }\ControlFlowTok{in}\NormalTok{ sigma\_frame)\{}
  \FunctionTok{set.seed}\NormalTok{(}\DecValTok{100}\NormalTok{)}
\NormalTok{  svm }\OtherTok{\textless{}{-}} \FunctionTok{ksvm}\NormalTok{(}\AttributeTok{x=}\NormalTok{x,}\AttributeTok{y=}\NormalTok{y,}\AttributeTok{data=}\NormalTok{sinData,}\AttributeTok{kernel=}\StringTok{"rbfdot"}\NormalTok{,}\AttributeTok{kpar=}\FunctionTok{list}\NormalTok{(}\AttributeTok{sigma=}\NormalTok{i),}\AttributeTok{C=}\DecValTok{1}\NormalTok{,}\AttributeTok{epsilon=}\FloatTok{0.1}\NormalTok{)}
\NormalTok{  svm.p}\OtherTok{\textless{}{-}}\FunctionTok{predict}\NormalTok{(svm,}\AttributeTok{newdata=}\NormalTok{x\_test)}
\NormalTok{  this.result }\OtherTok{\textless{}{-}} \FunctionTok{data.frame}\NormalTok{(i,}\FunctionTok{RMSE}\NormalTok{(svm.p,y\_test}\SpecialCharTok{$}\NormalTok{x))}
\NormalTok{  results }\OtherTok{=} \FunctionTok{rbind}\NormalTok{(results,this.result)}
\NormalTok{\}}

\FunctionTok{names}\NormalTok{(results) }\OtherTok{=} \FunctionTok{c}\NormalTok{(}\StringTok{"Sigma"}\NormalTok{,}\StringTok{"RMSE"}\NormalTok{)}

\FunctionTok{plot}\NormalTok{(results,}\AttributeTok{type=}\StringTok{"o"}\NormalTok{,}\AttributeTok{main=}\StringTok{"Sigma vs. RMSE"}\NormalTok{)}
\end{Highlighting}
\end{Shaded}

\includegraphics{mod4exercises_files/figure-latex/unnamed-chunk-6-1.pdf}
Overall, it seems that increasing sigma beyond 1 tends to produce an
increase in error rates,with the optimal value lying between zero and
one. Similarly, increasing the epsilon parameter increases the threshold
at which data points become relevant to modeling given that SVM is
epsilon-insensitive. Using an epsilon that is too high makes the model
dependent on points with radically low fit, this failing to generalize
to outside data.

\begin{enumerate}
\def\labelenumi{\arabic{enumi}.}
\setcounter{enumi}{1}
\tightlist
\item
  (15 points) For the Tecator data described in chapter 6 of the
  textbook, build SVM, neural network, MARS, and KNN models. Since
  neural networks are especially sensitive to highly correlated
  predictors, does pre-processing using PCA help the model? For this
  problem, you will perform your analysis in R.
\end{enumerate}

Because the textbook frames the data question as modeling the
relationship to fat, we use column 2 of the endpoints set.

\begin{Shaded}
\begin{Highlighting}[]
\CommentTok{\#load data {-} absorp is the predictor set and endpoints are the response set(s).}
\FunctionTok{library}\NormalTok{(caret)}
\FunctionTok{data}\NormalTok{(tecator)}
\FunctionTok{require}\NormalTok{(nnet)}
\end{Highlighting}
\end{Shaded}

\begin{verbatim}
## Loading required package: nnet
\end{verbatim}

\begin{Shaded}
\begin{Highlighting}[]
\FunctionTok{require}\NormalTok{(earth)}
\end{Highlighting}
\end{Shaded}

\begin{verbatim}
## Loading required package: earth
\end{verbatim}

\begin{verbatim}
## Warning: package 'earth' was built under R version 4.0.5
\end{verbatim}

\begin{verbatim}
## Loading required package: Formula
\end{verbatim}

\begin{verbatim}
## Loading required package: plotmo
\end{verbatim}

\begin{verbatim}
## Warning: package 'plotmo' was built under R version 4.0.5
\end{verbatim}

\begin{verbatim}
## Loading required package: plotrix
\end{verbatim}

\begin{verbatim}
## Loading required package: TeachingDemos
\end{verbatim}

\begin{verbatim}
## Warning: package 'TeachingDemos' was built under R version 4.0.5
\end{verbatim}

\begin{Shaded}
\begin{Highlighting}[]
\CommentTok{\#without PCA }
\NormalTok{x }\OtherTok{=} \FunctionTok{as.data.frame}\NormalTok{(absorp)}
\NormalTok{y }\OtherTok{=} \FunctionTok{as.data.frame}\NormalTok{(endpoints[,}\DecValTok{2}\NormalTok{])}
\NormalTok{x.p }\OtherTok{=} \FunctionTok{preProcess}\NormalTok{(x,}\AttributeTok{method =} \FunctionTok{c}\NormalTok{(}\StringTok{"center"}\NormalTok{,}\StringTok{"scale"}\NormalTok{))}
\NormalTok{x }\OtherTok{=} \FunctionTok{predict}\NormalTok{(x.p,x)}

\FunctionTok{names}\NormalTok{(y) }\OtherTok{=} \StringTok{\textquotesingle{}fat\textquotesingle{}}

\FunctionTok{set.seed}\NormalTok{(}\DecValTok{1}\NormalTok{)}
\NormalTok{trainingRows }\OtherTok{\textless{}{-}} \FunctionTok{createDataPartition}\NormalTok{(y}\SpecialCharTok{$}\NormalTok{fat,}\AttributeTok{p =} \FloatTok{0.80}\NormalTok{, }\AttributeTok{list=}\ConstantTok{FALSE}\NormalTok{)}

\NormalTok{trainx}\OtherTok{\textless{}{-}}\NormalTok{ x[trainingRows,]}
\NormalTok{trainy}\OtherTok{\textless{}{-}}\NormalTok{ y[trainingRows,]}
\NormalTok{testx }\OtherTok{\textless{}{-}}\NormalTok{ x[}\SpecialCharTok{{-}}\NormalTok{trainingRows,]}
\NormalTok{testy }\OtherTok{\textless{}{-}}\NormalTok{ y[}\SpecialCharTok{{-}}\NormalTok{trainingRows,]}



\CommentTok{\#SVM}
\FunctionTok{set.seed}\NormalTok{(}\DecValTok{100}\NormalTok{)}
\NormalTok{svmRtuned }\OtherTok{\textless{}{-}} \FunctionTok{train}\NormalTok{((trainx),(trainy),}\AttributeTok{method=}\StringTok{"svmRadial"}\NormalTok{,}\AttributeTok{preProc =} \FunctionTok{c}\NormalTok{(}\StringTok{"center"}\NormalTok{,}\StringTok{"scale"}\NormalTok{),}\AttributeTok{tunelength=}\DecValTok{14}\NormalTok{,}
                   \AttributeTok{trControl =} \FunctionTok{trainControl}\NormalTok{(}\AttributeTok{method=}\StringTok{"cv"}\NormalTok{))}
\NormalTok{svm.p }\OtherTok{\textless{}{-}} \FunctionTok{predict}\NormalTok{(svmRtuned}\SpecialCharTok{$}\NormalTok{finalModel,testx)}
\NormalTok{svm.rmse}\OtherTok{\textless{}{-}}\FunctionTok{RMSE}\NormalTok{(svm.p,testy)}
\FunctionTok{print}\NormalTok{(}\FunctionTok{paste}\NormalTok{(}\StringTok{"SVM"}\NormalTok{,svm.rmse))}
\end{Highlighting}
\end{Shaded}

\begin{verbatim}
## [1] "SVM 6.55562134092721"
\end{verbatim}

\begin{Shaded}
\begin{Highlighting}[]
\CommentTok{\#NN}
\FunctionTok{set.seed}\NormalTok{(}\DecValTok{100}\NormalTok{)}
\NormalTok{nnet }\OtherTok{\textless{}{-}} \FunctionTok{avNNet}\NormalTok{(trainx,trainy,}\AttributeTok{size=}\DecValTok{5}\NormalTok{,}\AttributeTok{decay=}\FloatTok{0.01}\NormalTok{,}\AttributeTok{repeats=}\DecValTok{5}\NormalTok{,}\AttributeTok{linout=}\ConstantTok{TRUE}\NormalTok{,}\AttributeTok{trace=}\ConstantTok{FALSE}\NormalTok{,}\AttributeTok{maxit=}\DecValTok{500}\NormalTok{,}\AttributeTok{maxNwts=}\DecValTok{5}\SpecialCharTok{*}\NormalTok{(}\FunctionTok{ncol}\NormalTok{(trainx)}\SpecialCharTok{+}\DecValTok{1}\NormalTok{)}\SpecialCharTok{+}\DecValTok{5}\SpecialCharTok{+}\DecValTok{1}\NormalTok{)}
\end{Highlighting}
\end{Shaded}

\begin{verbatim}
## Warning: executing %dopar% sequentially: no parallel backend registered
\end{verbatim}

\begin{Shaded}
\begin{Highlighting}[]
\NormalTok{nnet.p}\OtherTok{\textless{}{-}}\FunctionTok{predict}\NormalTok{(nnet,testx)}
\NormalTok{nnet.rmse}\OtherTok{\textless{}{-}}\FunctionTok{RMSE}\NormalTok{(nnet.p,testy)}
\FunctionTok{print}\NormalTok{(}\FunctionTok{paste}\NormalTok{(}\StringTok{"nnet"}\NormalTok{,nnet.rmse))}
\end{Highlighting}
\end{Shaded}

\begin{verbatim}
## [1] "nnet 0.37990906080454"
\end{verbatim}

\begin{Shaded}
\begin{Highlighting}[]
\CommentTok{\#MARS}
\NormalTok{marsgrid }\OtherTok{\textless{}{-}} \FunctionTok{expand.grid}\NormalTok{(}\AttributeTok{.degree=}\DecValTok{1}\SpecialCharTok{:}\DecValTok{2}\NormalTok{,}\AttributeTok{.nprune=}\DecValTok{2}\SpecialCharTok{:}\DecValTok{38}\NormalTok{)}
\FunctionTok{set.seed}\NormalTok{(}\DecValTok{100}\NormalTok{)}
\NormalTok{mars.t }\OtherTok{\textless{}{-}} \FunctionTok{train}\NormalTok{(trainx,trainy,}\AttributeTok{method=}\StringTok{"earth"}\NormalTok{,}\AttributeTok{tuneGrid=}\NormalTok{marsgrid,}\AttributeTok{trControl=}\FunctionTok{trainControl}\NormalTok{(}\AttributeTok{method=}\StringTok{"cv"}\NormalTok{))}
\NormalTok{mars.p }\OtherTok{\textless{}{-}} \FunctionTok{predict}\NormalTok{(mars.t,testx)}
\NormalTok{mars.rmse}\OtherTok{\textless{}{-}}\FunctionTok{RMSE}\NormalTok{(mars.p,testy)}
\FunctionTok{print}\NormalTok{(}\FunctionTok{paste}\NormalTok{(}\StringTok{"MARS"}\NormalTok{,mars.rmse))}
\end{Highlighting}
\end{Shaded}

\begin{verbatim}
## [1] "MARS 1.78792232873288"
\end{verbatim}

\begin{Shaded}
\begin{Highlighting}[]
\CommentTok{\#KNN {-} remove unbalanced first.}
\NormalTok{knnDescr }\OtherTok{\textless{}{-}}\NormalTok{ trainx[,}\SpecialCharTok{{-}}\FunctionTok{nearZeroVar}\NormalTok{(trainx)]}
\NormalTok{knntune  }\OtherTok{\textless{}{-}} \FunctionTok{train}\NormalTok{(knnDescr,trainy,}\AttributeTok{method=}\StringTok{"knn"}\NormalTok{,}\AttributeTok{tuneGrid=}\FunctionTok{data.frame}\NormalTok{(}\AttributeTok{.k=}\DecValTok{1}\SpecialCharTok{:}\DecValTok{20}\NormalTok{),}\AttributeTok{trControl=}\FunctionTok{trainControl}\NormalTok{(}\AttributeTok{method=}\StringTok{"cv"}\NormalTok{))}
\end{Highlighting}
\end{Shaded}

\begin{verbatim}
## Warning in nominalTrainWorkflow(x = x, y = y, wts = weights, info = trainInfo, :
## There were missing values in resampled performance measures.
\end{verbatim}

\begin{Shaded}
\begin{Highlighting}[]
\NormalTok{knn.p }\OtherTok{\textless{}{-}}\FunctionTok{predict}\NormalTok{(knntune,testx)}
\NormalTok{knn.rmse}\OtherTok{\textless{}{-}}\FunctionTok{RMSE}\NormalTok{(knn.p,testy)}
\FunctionTok{print}\NormalTok{(}\FunctionTok{paste}\NormalTok{(}\StringTok{"KNN"}\NormalTok{,knn.rmse))}
\end{Highlighting}
\end{Shaded}

\begin{verbatim}
## [1] "KNN 12.4431471959476"
\end{verbatim}

\begin{Shaded}
\begin{Highlighting}[]
\CommentTok{\#NN with PCA }
\NormalTok{x }\OtherTok{=} \FunctionTok{as.data.frame}\NormalTok{(absorp)}
\NormalTok{y }\OtherTok{=} \FunctionTok{as.data.frame}\NormalTok{(endpoints[,}\DecValTok{2}\NormalTok{])}
\NormalTok{x.p }\OtherTok{=} \FunctionTok{preProcess}\NormalTok{(x,}\AttributeTok{method =} \FunctionTok{c}\NormalTok{(}\StringTok{"center"}\NormalTok{,}\StringTok{"scale"}\NormalTok{,}\StringTok{"pca"}\NormalTok{))}
\NormalTok{x }\OtherTok{=} \FunctionTok{predict}\NormalTok{(x.p,x)}

\FunctionTok{names}\NormalTok{(y) }\OtherTok{=} \StringTok{\textquotesingle{}fat\textquotesingle{}}

\FunctionTok{set.seed}\NormalTok{(}\DecValTok{1}\NormalTok{)}
\NormalTok{trainingRows }\OtherTok{\textless{}{-}} \FunctionTok{createDataPartition}\NormalTok{(y}\SpecialCharTok{$}\NormalTok{fat,}\AttributeTok{p =} \FloatTok{0.80}\NormalTok{, }\AttributeTok{list=}\ConstantTok{FALSE}\NormalTok{)}

\NormalTok{trainx}\OtherTok{\textless{}{-}}\NormalTok{ x[trainingRows,]}
\NormalTok{trainy}\OtherTok{\textless{}{-}}\NormalTok{ y[trainingRows,]}
\NormalTok{testx }\OtherTok{\textless{}{-}}\NormalTok{ x[}\SpecialCharTok{{-}}\NormalTok{trainingRows,]}
\NormalTok{testy }\OtherTok{\textless{}{-}}\NormalTok{ y[}\SpecialCharTok{{-}}\NormalTok{trainingRows,]}

\CommentTok{\#NN}
\FunctionTok{set.seed}\NormalTok{(}\DecValTok{100}\NormalTok{)}
\NormalTok{nnet }\OtherTok{\textless{}{-}} \FunctionTok{avNNet}\NormalTok{(trainx,trainy,}\AttributeTok{size=}\DecValTok{5}\NormalTok{,}\AttributeTok{decay=}\FloatTok{0.01}\NormalTok{,}\AttributeTok{repeats=}\DecValTok{5}\NormalTok{,}\AttributeTok{linout=}\ConstantTok{TRUE}\NormalTok{,}\AttributeTok{trace=}\ConstantTok{FALSE}\NormalTok{,}\AttributeTok{maxit=}\DecValTok{500}\NormalTok{,}\AttributeTok{maxNwts=}\DecValTok{5}\SpecialCharTok{*}\NormalTok{(}\FunctionTok{ncol}\NormalTok{(trainx)}\SpecialCharTok{+}\DecValTok{1}\NormalTok{)}\SpecialCharTok{+}\DecValTok{5}\SpecialCharTok{+}\DecValTok{1}\NormalTok{)}
\NormalTok{nnet.p}\OtherTok{\textless{}{-}}\FunctionTok{predict}\NormalTok{(nnet,testx)}
\NormalTok{nnet.rmse}\OtherTok{\textless{}{-}}\FunctionTok{RMSE}\NormalTok{(nnet.p,testy)}
\FunctionTok{print}\NormalTok{(}\FunctionTok{paste}\NormalTok{(}\StringTok{"nnet w PCA"}\NormalTok{,nnet.rmse))}
\end{Highlighting}
\end{Shaded}

\begin{verbatim}
## [1] "nnet w PCA 11.1405942861103"
\end{verbatim}

PCA failed to improve our neural net.

\begin{enumerate}
\def\labelenumi{\arabic{enumi})}
\setcounter{enumi}{2}
\tightlist
\item
  The matrix processPredictors contains the 57 predictors (12 describing
  the input biological material and 45 describing the process
  predictors) for the 176 manufacturing runs. yield contains the percent
  yield for each run. This describes the data for a chemical
  manufacturing process. Use data imputation, data splitting, and
  pre-processing steps and train several nonlinear regression models.
\end{enumerate}

\begin{enumerate}
\def\labelenumi{(\alph{enumi})}
\setcounter{enumi}{1}
\tightlist
\item
  Which predictors are most important in the optimal nonlinear
  regression model? Do either the biological or process variables
  dominate the list? How do the top ten important predictors compare to
  the top ten predictors from the optimal linear model?
\item
  Explore the relationships between the top predictors and the response
  for the predictors that are unique to the optimal nonlinear regression
  model. Do these plots reveal intuition about the biological or process
  predictors and their relationship with yield?
\end{enumerate}

For this problem, you will perform your analysis in R.

\begin{Shaded}
\begin{Highlighting}[]
\CommentTok{\#load data}
\FunctionTok{library}\NormalTok{(AppliedPredictiveModeling)}
\end{Highlighting}
\end{Shaded}

\begin{verbatim}
## Warning: package 'AppliedPredictiveModeling' was built under R version 4.0.5
\end{verbatim}

\begin{Shaded}
\begin{Highlighting}[]
\FunctionTok{data}\NormalTok{(ChemicalManufacturingProcess)}

\CommentTok{\#knn imputation}
\FunctionTok{require}\NormalTok{(Hmisc)}
\end{Highlighting}
\end{Shaded}

\begin{verbatim}
## Loading required package: Hmisc
\end{verbatim}

\begin{verbatim}
## Warning: package 'Hmisc' was built under R version 4.0.5
\end{verbatim}

\begin{verbatim}
## Loading required package: survival
\end{verbatim}

\begin{verbatim}
## 
## Attaching package: 'survival'
\end{verbatim}

\begin{verbatim}
## The following object is masked from 'package:caret':
## 
##     cluster
\end{verbatim}

\begin{verbatim}
## 
## Attaching package: 'Hmisc'
\end{verbatim}

\begin{verbatim}
## The following objects are masked from 'package:TeachingDemos':
## 
##     cnvrt.coords, subplot
\end{verbatim}

\begin{verbatim}
## The following objects are masked from 'package:dplyr':
## 
##     src, summarize
\end{verbatim}

\begin{verbatim}
## The following objects are masked from 'package:base':
## 
##     format.pval, units
\end{verbatim}

\begin{Shaded}
\begin{Highlighting}[]
\NormalTok{x }\OtherTok{=}\NormalTok{ (ChemicalManufacturingProcess)}
\NormalTok{x }\OtherTok{=} \FunctionTok{subset}\NormalTok{(x,}\AttributeTok{select=}\SpecialCharTok{{-}}\FunctionTok{c}\NormalTok{(Yield))}
\NormalTok{x }\OtherTok{=} \FunctionTok{as.data.frame}\NormalTok{(x)}
\NormalTok{x }\OtherTok{=}\NormalTok{ Hmisc}\SpecialCharTok{::}\FunctionTok{impute}\NormalTok{(x)}
\NormalTok{y  }\OtherTok{\textless{}{-}}\NormalTok{ChemicalManufacturingProcess}\SpecialCharTok{$}\NormalTok{Yield }\SpecialCharTok{\%\textgreater{}\%}\NormalTok{ as.data.frame }

\CommentTok{\#remove nzv }
\NormalTok{nzv }\OtherTok{=} \FunctionTok{nearZeroVar}\NormalTok{(x)}
\NormalTok{x }\OtherTok{=}\NormalTok{ x[}\SpecialCharTok{{-}}\NormalTok{nzv]}

\CommentTok{\#center, scale }
\NormalTok{x.cs }\OtherTok{\textless{}{-}} \FunctionTok{preProcess}\NormalTok{(x,}\AttributeTok{method=}\FunctionTok{c}\NormalTok{(}\StringTok{"center"}\NormalTok{,}\StringTok{"scale"}\NormalTok{))}
\NormalTok{x }\OtherTok{=} \FunctionTok{predict}\NormalTok{(x.cs,x)}

\CommentTok{\#partition}
\FunctionTok{set.seed}\NormalTok{(}\DecValTok{1}\NormalTok{)}
\FunctionTok{names}\NormalTok{(y)}\OtherTok{=}\StringTok{"yield"}
\NormalTok{trainingRows }\OtherTok{\textless{}{-}} \FunctionTok{createDataPartition}\NormalTok{(y}\SpecialCharTok{$}\NormalTok{yield,}\AttributeTok{p =} \FloatTok{0.80}\NormalTok{, }\AttributeTok{list=}\ConstantTok{FALSE}\NormalTok{)}

\NormalTok{trainx}\OtherTok{\textless{}{-}}\NormalTok{ x[trainingRows,]}
\NormalTok{trainy}\OtherTok{\textless{}{-}}\NormalTok{ y[trainingRows,]}
\NormalTok{testx }\OtherTok{\textless{}{-}}\NormalTok{ x[}\SpecialCharTok{{-}}\NormalTok{trainingRows,]}
\NormalTok{testy }\OtherTok{\textless{}{-}}\NormalTok{ y[}\SpecialCharTok{{-}}\NormalTok{trainingRows,]}
\end{Highlighting}
\end{Shaded}

\begin{enumerate}
\def\labelenumi{(\alph{enumi})}
\tightlist
\item
  Which nonlinear regression model gives the optimal resampling and test
  set performance? For this instance, we will use svm, nnet, and mars.
\end{enumerate}

\begin{Shaded}
\begin{Highlighting}[]
\CommentTok{\#SVM}
\FunctionTok{set.seed}\NormalTok{(}\DecValTok{100}\NormalTok{)}
\NormalTok{svmRtuned }\OtherTok{\textless{}{-}} \FunctionTok{train}\NormalTok{((trainx),(trainy),}\AttributeTok{method=}\StringTok{"svmRadial"}\NormalTok{,}\AttributeTok{preProc =} \FunctionTok{c}\NormalTok{(}\StringTok{"center"}\NormalTok{,}\StringTok{"scale"}\NormalTok{),}\AttributeTok{tunelength=}\DecValTok{14}\NormalTok{,}
                   \AttributeTok{trControl =} \FunctionTok{trainControl}\NormalTok{(}\AttributeTok{method=}\StringTok{"cv"}\NormalTok{))}
\NormalTok{svm.p }\OtherTok{\textless{}{-}} \FunctionTok{predict}\NormalTok{(svmRtuned}\SpecialCharTok{$}\NormalTok{finalModel,testx)}
\NormalTok{svm.rmse}\OtherTok{\textless{}{-}}\FunctionTok{RMSE}\NormalTok{(svm.p,testy)}

\CommentTok{\#NN}
\FunctionTok{set.seed}\NormalTok{(}\DecValTok{100}\NormalTok{)}
\NormalTok{nnet }\OtherTok{\textless{}{-}} \FunctionTok{avNNet}\NormalTok{(trainx,trainy,}\AttributeTok{size=}\DecValTok{5}\NormalTok{,}\AttributeTok{decay=}\FloatTok{0.01}\NormalTok{,}\AttributeTok{repeats=}\DecValTok{5}\NormalTok{,}\AttributeTok{linout=}\ConstantTok{TRUE}\NormalTok{,}\AttributeTok{trace=}\ConstantTok{FALSE}\NormalTok{,}\AttributeTok{maxit=}\DecValTok{500}\NormalTok{,}\AttributeTok{maxNwts=}\DecValTok{5}\SpecialCharTok{*}\NormalTok{(}\FunctionTok{ncol}\NormalTok{(trainx)}\SpecialCharTok{+}\DecValTok{1}\NormalTok{)}\SpecialCharTok{+}\DecValTok{5}\SpecialCharTok{+}\DecValTok{1}\NormalTok{)}
\NormalTok{nnet.p}\OtherTok{\textless{}{-}}\FunctionTok{predict}\NormalTok{(nnet,testx)}
\NormalTok{nnet.rmse}\OtherTok{\textless{}{-}}\FunctionTok{RMSE}\NormalTok{(nnet.p,testy)}

\CommentTok{\#MARS}
\NormalTok{marsgrid }\OtherTok{\textless{}{-}} \FunctionTok{expand.grid}\NormalTok{(}\AttributeTok{.degree=}\DecValTok{1}\SpecialCharTok{:}\DecValTok{2}\NormalTok{,}\AttributeTok{.nprune=}\DecValTok{2}\SpecialCharTok{:}\DecValTok{38}\NormalTok{)}
\FunctionTok{set.seed}\NormalTok{(}\DecValTok{100}\NormalTok{)}
\NormalTok{mars.t }\OtherTok{\textless{}{-}} \FunctionTok{train}\NormalTok{(trainx,trainy,}\AttributeTok{method=}\StringTok{"earth"}\NormalTok{,}\AttributeTok{tuneGrid=}\NormalTok{marsgrid,}\AttributeTok{trControl=}\FunctionTok{trainControl}\NormalTok{(}\AttributeTok{method=}\StringTok{"cv"}\NormalTok{))}
\NormalTok{mars.p }\OtherTok{\textless{}{-}} \FunctionTok{predict}\NormalTok{(mars.t,testx)}
\NormalTok{mars.rmse}\OtherTok{\textless{}{-}}\FunctionTok{RMSE}\NormalTok{(mars.p,testy)}


\FunctionTok{print}\NormalTok{(}\FunctionTok{paste}\NormalTok{(}\StringTok{"SVM"}\NormalTok{,svm.rmse))}
\end{Highlighting}
\end{Shaded}

\begin{verbatim}
## [1] "SVM 0.984948379124365"
\end{verbatim}

\begin{Shaded}
\begin{Highlighting}[]
\FunctionTok{print}\NormalTok{(}\FunctionTok{paste}\NormalTok{(}\StringTok{"MARS"}\NormalTok{,mars.rmse))}
\end{Highlighting}
\end{Shaded}

\begin{verbatim}
## [1] "MARS 1.04743569857994"
\end{verbatim}

\begin{Shaded}
\begin{Highlighting}[]
\FunctionTok{print}\NormalTok{(}\FunctionTok{paste}\NormalTok{(}\StringTok{"nnet"}\NormalTok{,nnet.rmse))}
\end{Highlighting}
\end{Shaded}

\begin{verbatim}
## [1] "nnet 0.996321573653602"
\end{verbatim}

Our SVM model performed the best with a test RMSE of 0.984 and the
following training RMSE of 1.21:

\begin{Shaded}
\begin{Highlighting}[]
\FunctionTok{print}\NormalTok{(svmRtuned)}
\end{Highlighting}
\end{Shaded}

\begin{verbatim}
## Support Vector Machines with Radial Basis Function Kernel 
## 
## 144 samples
##  56 predictor
## 
## Pre-processing: centered (56), scaled (56) 
## Resampling: Cross-Validated (10 fold) 
## Summary of sample sizes: 129, 130, 130, 130, 130, 130, ... 
## Resampling results across tuning parameters:
## 
##   C     RMSE      Rsquared   MAE      
##   0.25  1.452999  0.4649044  1.1709934
##   0.50  1.340040  0.5278748  1.0673606
##   1.00  1.212608  0.6005815  0.9576347
## 
## Tuning parameter 'sigma' was held constant at a value of 0.02358453
## RMSE was used to select the optimal model using the smallest value.
## The final values used for the model were sigma = 0.02358453 and C = 1.
\end{verbatim}

\begin{enumerate}
\def\labelenumi{(\alph{enumi})}
\setcounter{enumi}{1}
\tightlist
\item
  Which predictors are most important in the optimal nonlinear
  regression model? Do either the biological or process variables
  dominate the list? How do the top ten important predictors compare to
  the top ten predictors from the optimal linear model?
\end{enumerate}

\begin{Shaded}
\begin{Highlighting}[]
\NormalTok{imp }\OtherTok{=} \FunctionTok{varImp}\NormalTok{(svmRtuned)}\SpecialCharTok{$}\NormalTok{importance}
\end{Highlighting}
\end{Shaded}

The most important predictors are (in order): ManufacturingProcess32
ManufacturingProcess13 BiologicalMaterial06 BiologicalMaterial03
ManufacturingProcess17 BiologicalMaterial12 ManufacturingProcess09
BiologicalMaterial02 ManufacturingProcess06 ManufacturingProcess31

Manufacturing appears to dominate.

\begin{Shaded}
\begin{Highlighting}[]
\CommentTok{\#construct an optimal linear model}
\FunctionTok{library}\NormalTok{(glmnet)}
\end{Highlighting}
\end{Shaded}

\begin{verbatim}
## Warning: package 'glmnet' was built under R version 4.0.5
\end{verbatim}

\begin{verbatim}
## Loading required package: Matrix
\end{verbatim}

\begin{verbatim}
## Loaded glmnet 4.1-1
\end{verbatim}

\begin{Shaded}
\begin{Highlighting}[]
\FunctionTok{require}\NormalTok{(caret)}
\CommentTok{\#perform k{-}fold cross{-}validation to find optimal lambda value}
\NormalTok{cv\_model }\OtherTok{\textless{}{-}} \FunctionTok{cv.glmnet}\NormalTok{(}\FunctionTok{as.matrix}\NormalTok{(trainx), trainy, }\AttributeTok{alpha =} \FloatTok{0.5}\NormalTok{)}

\CommentTok{\#find optimal lambda value that minimizes test MSE}
\NormalTok{best\_lambda }\OtherTok{\textless{}{-}}\NormalTok{ cv\_model}\SpecialCharTok{$}\NormalTok{lambda.min}
\NormalTok{best\_model }\OtherTok{\textless{}{-}} \FunctionTok{glmnet}\NormalTok{(}\FunctionTok{as.matrix}\NormalTok{(trainx), trainy, }\AttributeTok{alpha =} \FloatTok{0.5}\NormalTok{, }\AttributeTok{lambda =}\NormalTok{ best\_lambda)}
\NormalTok{lasso }\OtherTok{\textless{}{-}} \FunctionTok{predict}\NormalTok{(best\_model, }\AttributeTok{s =}\NormalTok{ best\_lambda, }\AttributeTok{newx =} \FunctionTok{as.matrix}\NormalTok{(testx))}
\NormalTok{lasso.rmse }\OtherTok{\textless{}{-}} \FunctionTok{RMSE}\NormalTok{(lasso,testy)}

\NormalTok{lasso.imp }\OtherTok{=} \FunctionTok{varImp}\NormalTok{(best\_model,best\_model}\SpecialCharTok{$}\NormalTok{lambda) }\SpecialCharTok{\%\textgreater{}\%} \FunctionTok{subset}\NormalTok{(Overall }\SpecialCharTok{\textgreater{}} \DecValTok{0}\NormalTok{) }
\NormalTok{lasso.imp }\OtherTok{=} 
\FunctionTok{print}\NormalTok{(}\FunctionTok{head}\NormalTok{(lasso.imp,}\AttributeTok{n=}\DecValTok{10}\NormalTok{))}
\end{Highlighting}
\end{Shaded}

\begin{verbatim}
##                           Overall
## BiologicalMaterial03   0.01423861
## BiologicalMaterial06   0.09731297
## ManufacturingProcess09 0.36042005
## ManufacturingProcess13 0.21886294
## ManufacturingProcess15 0.01977422
## ManufacturingProcess17 0.21249807
## ManufacturingProcess32 0.81874937
## ManufacturingProcess33 0.04338854
## ManufacturingProcess36 0.01087368
## ManufacturingProcess37 0.04999779
\end{verbatim}

In the instance of our Lasso model, manufacturing also dominates but at
a slightly higher rate.

\begin{enumerate}
\def\labelenumi{(\alph{enumi})}
\setcounter{enumi}{2}
\tightlist
\item
  Explore the relationships between the top predictors and the response
  for the predictors that are unique to the optimal nonlinear regression
  model. Do these plots reveal intuition about the biological or process
  predictors and their relationship with yield?
\end{enumerate}

Unique predictors to the nonlinear model:

BiologicalMaterial12 ManufacturingProcess09 BiologicalMaterial02
ManufacturingProcess06 ManufacturingProcess31

\begin{Shaded}
\begin{Highlighting}[]
\NormalTok{unique.p }\OtherTok{=} \FunctionTok{subset}\NormalTok{(trainx,}\AttributeTok{select=}\FunctionTok{c}\NormalTok{(BiologicalMaterial12,ManufacturingProcess09,BiologicalMaterial02,ManufacturingProcess06,ManufacturingProcess31))}
\NormalTok{unique.p}\SpecialCharTok{$}\NormalTok{yield }\OtherTok{=}\NormalTok{ trainy}

\FunctionTok{cor}\NormalTok{(unique.p)}
\end{Highlighting}
\end{Shaded}

\begin{verbatim}
##                        BiologicalMaterial12 ManufacturingProcess09
## BiologicalMaterial12             1.00000000              0.2211858
## ManufacturingProcess09           0.22118585              1.0000000
## BiologicalMaterial02             0.78948338              0.1677067
## ManufacturingProcess06           0.14128234              0.2151233
## ManufacturingProcess31          -0.04100431             -0.2522044
## yield                            0.35369877              0.4775427
##                        BiologicalMaterial02 ManufacturingProcess06
## BiologicalMaterial12             0.78948338             0.14128234
## ManufacturingProcess09           0.16770673             0.21512334
## BiologicalMaterial02             1.00000000             0.12964602
## ManufacturingProcess06           0.12964602             1.00000000
## ManufacturingProcess31          -0.09333995            -0.03050484
## yield                            0.44517169             0.14195375
##                        ManufacturingProcess31        yield
## BiologicalMaterial12             -0.041004311  0.353698768
## ManufacturingProcess09           -0.252204422  0.477542656
## BiologicalMaterial02             -0.093339953  0.445171693
## ManufacturingProcess06           -0.030504840  0.141953750
## ManufacturingProcess31            1.000000000 -0.006102568
## yield                            -0.006102568  1.000000000
\end{verbatim}

None of the unique predictors seem to suggest any strong visible
correlation with the Yield variable separately, with the highest
correlation only going into the 40th percentile.

\begin{enumerate}
\def\labelenumi{\arabic{enumi}.}
\setcounter{enumi}{3}
\tightlist
\item
  (15 points) Recreate the simulated data as shown below:
\end{enumerate}

\begin{enumerate}
\def\labelenumi{(\alph{enumi})}
\tightlist
\item
  Fit a random forest model to all of the predictors, then estimate the
  variable importance scores:
\end{enumerate}

Did the random forest model significantly use the uninformative
predictors (V6 -- V10)?

For this problem, you will perform your analysis in R.

\begin{Shaded}
\begin{Highlighting}[]
\FunctionTok{library}\NormalTok{(mlbench)}
\end{Highlighting}
\end{Shaded}

\begin{verbatim}
## Warning: package 'mlbench' was built under R version 4.0.5
\end{verbatim}

\begin{Shaded}
\begin{Highlighting}[]
\FunctionTok{set.seed}\NormalTok{(}\DecValTok{200}\NormalTok{)}
\NormalTok{simulated }\OtherTok{\textless{}{-}} \FunctionTok{mlbench.friedman1}\NormalTok{(}\DecValTok{200}\NormalTok{, }\AttributeTok{sd =} \DecValTok{1}\NormalTok{) }
\NormalTok{simulated }\OtherTok{\textless{}{-}} \FunctionTok{cbind}\NormalTok{(simulated}\SpecialCharTok{$}\NormalTok{x, simulated}\SpecialCharTok{$}\NormalTok{y)}
\NormalTok{simulated }\OtherTok{\textless{}{-}} \FunctionTok{as.data.frame}\NormalTok{(simulated)}
\FunctionTok{colnames}\NormalTok{(simulated)[}\FunctionTok{ncol}\NormalTok{(simulated)] }\OtherTok{\textless{}{-}} \StringTok{"y"}

\FunctionTok{library}\NormalTok{(randomForest)}
\end{Highlighting}
\end{Shaded}

\begin{verbatim}
## randomForest 4.6-14
\end{verbatim}

\begin{verbatim}
## Type rfNews() to see new features/changes/bug fixes.
\end{verbatim}

\begin{verbatim}
## 
## Attaching package: 'randomForest'
\end{verbatim}

\begin{verbatim}
## The following object is masked from 'package:dplyr':
## 
##     combine
\end{verbatim}

\begin{verbatim}
## The following object is masked from 'package:ggplot2':
## 
##     margin
\end{verbatim}

\begin{Shaded}
\begin{Highlighting}[]
\FunctionTok{library}\NormalTok{(caret)}
\NormalTok{model1 }\OtherTok{\textless{}{-}} \FunctionTok{randomForest}\NormalTok{(y }\SpecialCharTok{\textasciitilde{}}\NormalTok{ ., }\AttributeTok{data =}\NormalTok{ simulated, }\AttributeTok{importance =} \ConstantTok{TRUE}\NormalTok{,}
\AttributeTok{ntree =} \DecValTok{1000}\NormalTok{) }
\NormalTok{rfImp1 }\OtherTok{\textless{}{-}} \FunctionTok{varImp}\NormalTok{(model1, }\AttributeTok{scale =} \ConstantTok{FALSE}\NormalTok{)}
\FunctionTok{print}\NormalTok{(rfImp1)}
\end{Highlighting}
\end{Shaded}

\begin{verbatim}
##          Overall
## V1   8.732235404
## V2   6.415369387
## V3   0.763591825
## V4   7.615118809
## V5   2.023524577
## V6   0.165111172
## V7  -0.005961659
## V8  -0.166362581
## V9  -0.095292651
## V10 -0.074944788
\end{verbatim}

The model did not significantly use V6-V10 compared to the other
predictors.

\begin{enumerate}
\def\labelenumi{(\alph{enumi})}
\setcounter{enumi}{1}
\tightlist
\item
  Now add an additional predictor that is highly correlated with one of
  the informative predictors. For example:
  simulated\(duplicate1 <- simulated\)V1 + rnorm(200) * .1
  \textgreater{} cor(simulated\(duplicate1, simulated\)V1) Fit another
  random forest model to these data. Did the importance score for V1
  change? What happens when you add another predictor that is also
  highly correlated with V1?
\end{enumerate}

\begin{Shaded}
\begin{Highlighting}[]
\NormalTok{simulated}\SpecialCharTok{$}\NormalTok{dup1 }\OtherTok{\textless{}{-}}\NormalTok{ simulated}\SpecialCharTok{$}\NormalTok{V1 }\SpecialCharTok{+} \FunctionTok{rnorm}\NormalTok{(}\DecValTok{200}\NormalTok{) }\SpecialCharTok{*} \FloatTok{0.1}
\CommentTok{\#sanity check}
\FunctionTok{cor}\NormalTok{(simulated}\SpecialCharTok{$}\NormalTok{V1,simulated}\SpecialCharTok{$}\NormalTok{dup1)}
\end{Highlighting}
\end{Shaded}

\begin{verbatim}
## [1] 0.9460206
\end{verbatim}

\begin{Shaded}
\begin{Highlighting}[]
\CommentTok{\#refit}
\NormalTok{model1 }\OtherTok{\textless{}{-}} \FunctionTok{randomForest}\NormalTok{(y }\SpecialCharTok{\textasciitilde{}}\NormalTok{ ., }\AttributeTok{data =}\NormalTok{ simulated, }\AttributeTok{importance =} \ConstantTok{TRUE}\NormalTok{,}
\AttributeTok{ntree =} \DecValTok{1000}\NormalTok{) }
\NormalTok{rfImp1 }\OtherTok{\textless{}{-}} \FunctionTok{varImp}\NormalTok{(model1, }\AttributeTok{scale =} \ConstantTok{FALSE}\NormalTok{)}

\FunctionTok{print}\NormalTok{(rfImp1)}
\end{Highlighting}
\end{Shaded}

\begin{verbatim}
##          Overall
## V1    5.69119973
## V2    6.06896061
## V3    0.62970218
## V4    7.04752238
## V5    1.87238438
## V6    0.13569065
## V7   -0.01345645
## V8   -0.04370565
## V9    0.00840438
## V10   0.02894814
## dup1  4.28331581
\end{verbatim}

V1 dropped in importance while the dup1 variable took on a similar
overall value. When we add another predictor, all three colinear
predictors drop in value, with V1 having the highest magnitude and Dup1
havingthe second highest.

\begin{Shaded}
\begin{Highlighting}[]
\NormalTok{simulated}\SpecialCharTok{$}\NormalTok{dup2 }\OtherTok{\textless{}{-}}\NormalTok{ simulated}\SpecialCharTok{$}\NormalTok{V1 }\SpecialCharTok{+} \FunctionTok{rnorm}\NormalTok{(}\DecValTok{200}\NormalTok{) }\SpecialCharTok{*} \FloatTok{0.1}
\CommentTok{\#refit}
\NormalTok{model1 }\OtherTok{\textless{}{-}} \FunctionTok{randomForest}\NormalTok{(y }\SpecialCharTok{\textasciitilde{}}\NormalTok{ ., }\AttributeTok{data =}\NormalTok{ simulated, }\AttributeTok{importance =} \ConstantTok{TRUE}\NormalTok{,}
\AttributeTok{ntree =} \DecValTok{1000}\NormalTok{) }
\NormalTok{rfImp1 }\OtherTok{\textless{}{-}} \FunctionTok{varImp}\NormalTok{(model1, }\AttributeTok{scale =} \ConstantTok{FALSE}\NormalTok{)}

\FunctionTok{print}\NormalTok{(rfImp1)}
\end{Highlighting}
\end{Shaded}

\begin{verbatim}
##          Overall
## V1    4.91687329
## V2    6.52816504
## V3    0.58711552
## V4    7.04870917
## V5    2.03115561
## V6    0.14213148
## V7    0.10991985
## V8   -0.08405687
## V9   -0.01075028
## V10   0.09230576
## dup1  3.80068234
## dup2  1.87721959
\end{verbatim}

\begin{enumerate}
\def\labelenumi{\arabic{enumi}.}
\setcounter{enumi}{4}
\tightlist
\item
  (15 points) Use a simulation to show tree bias with different
  granularities. For this problem, you will perform your analysis in R.
\end{enumerate}

\begin{Shaded}
\begin{Highlighting}[]
\FunctionTok{require}\NormalTok{(rpart)}
\end{Highlighting}
\end{Shaded}

\begin{verbatim}
## Loading required package: rpart
\end{verbatim}

\begin{Shaded}
\begin{Highlighting}[]
\FunctionTok{library}\NormalTok{(mlbench)}
\FunctionTok{set.seed}\NormalTok{(}\DecValTok{200}\NormalTok{)}
\NormalTok{simulated }\OtherTok{\textless{}{-}} \FunctionTok{mlbench.friedman1}\NormalTok{(}\DecValTok{300}\NormalTok{, }\AttributeTok{sd =} \DecValTok{1}\NormalTok{) }
\NormalTok{simulated }\OtherTok{\textless{}{-}} \FunctionTok{cbind}\NormalTok{(simulated}\SpecialCharTok{$}\NormalTok{x, simulated}\SpecialCharTok{$}\NormalTok{y)}
\NormalTok{simulated }\OtherTok{\textless{}{-}} \FunctionTok{as.data.frame}\NormalTok{(simulated)}
\FunctionTok{colnames}\NormalTok{(simulated)[}\FunctionTok{ncol}\NormalTok{(simulated)] }\OtherTok{\textless{}{-}} \StringTok{"y"}

\NormalTok{trainingRows }\OtherTok{\textless{}{-}} \FunctionTok{createDataPartition}\NormalTok{(simulated}\SpecialCharTok{$}\NormalTok{y,}\AttributeTok{p =} \FloatTok{0.80}\NormalTok{, }\AttributeTok{list=}\ConstantTok{FALSE}\NormalTok{)}
\NormalTok{train }\OtherTok{\textless{}{-}}\NormalTok{ simulated[trainingRows,]}
\NormalTok{test  }\OtherTok{\textless{}{-}}\NormalTok{ simulated[}\SpecialCharTok{{-}}\NormalTok{trainingRows,]}

\NormalTok{trees.rmse }\OtherTok{=} \FunctionTok{c}\NormalTok{()}

\ControlFlowTok{for}\NormalTok{ (i }\ControlFlowTok{in} \FunctionTok{c}\NormalTok{(}\DecValTok{1}\SpecialCharTok{:}\DecValTok{10}\NormalTok{))\{}
\NormalTok{  rpartTree }\OtherTok{\textless{}{-}} \FunctionTok{rpart}\NormalTok{(y }\SpecialCharTok{\textasciitilde{}}\NormalTok{ ., }\AttributeTok{data =} \FunctionTok{as.data.frame}\NormalTok{(train),}\AttributeTok{maxdepth=}\NormalTok{i)}
\NormalTok{  tree.p }\OtherTok{\textless{}{-}} \FunctionTok{predict}\NormalTok{(rpartTree,}\FunctionTok{subset}\NormalTok{(test,}\AttributeTok{select=}\SpecialCharTok{{-}}\FunctionTok{c}\NormalTok{(y)))}
\NormalTok{  trees.rmse }\OtherTok{\textless{}{-}} \FunctionTok{cbind}\NormalTok{(trees.rmse,}\FunctionTok{RMSE}\NormalTok{(tree.p,test}\SpecialCharTok{$}\NormalTok{y))}
\NormalTok{\}}

\FunctionTok{plot}\NormalTok{(}\AttributeTok{x=}\FunctionTok{c}\NormalTok{(}\DecValTok{1}\SpecialCharTok{:}\DecValTok{10}\NormalTok{),}\AttributeTok{y=}\NormalTok{trees.rmse,}\AttributeTok{main=}\StringTok{"Depth vs. RMSE"}\NormalTok{,}\AttributeTok{type=}\StringTok{"o"}\NormalTok{)}
\end{Highlighting}
\end{Shaded}

\includegraphics{mod4exercises_files/figure-latex/unnamed-chunk-19-1.pdf}
6. (15 points) In stochastic gradient boosting, the bagging fraction and
learning rate will govern the construction of the trees as they are
guided by the gradient. Although the optimal values of these parameters
should be obtained through the tuning process, it is helpful to
understand how the magnitudes of these parameters affect magnitudes of
variable importance. Figure 8.24 below provides the variable importance
plots for boosting using two extreme values for the bagging fraction
(0.1 and 0.9) and the learning rate (0.1 and 0.9) for the solubility
data. The left-hand plot has both parameters set to 0.1, and the
right-hand plot has both set to 0.9:

\begin{enumerate}
\def\labelenumi{(\alph{enumi})}
\tightlist
\item
  Why does the model on the right focus its importance on just the first
  few of predictors, whereas the model on the left spreads importance
  across more predictors?
\end{enumerate}

A higher learning rate means that the step size during our optimization
process is larger - generally, a larger learning rate means that fewer
iterations of tuning are required to construct a model. This, combined
with the fact that 90\% of the training set is used in each bagging
iteration, means that the model on the right is likely to pick up on the
same signal across samples and terminate faster, leading it to favor a
handful of predictors. The converse is true for the model on the left,
which takes longer to fully propagate due to its learning rate, and
because of the small bagging fraction, is likely to capture a variety of
different signals.

\begin{enumerate}
\def\labelenumi{(\alph{enumi})}
\setcounter{enumi}{1}
\item
  Which model do you think would be more predictive of other samples?
  The model on the left exhibits less bias the first few predictors, and
  is less likely to underfit unknown data.
\item
  How would increasing interaction depth affect the slope of predictor
  importance for either model in Fig. 8.24? A smaller learning rate
  requires more iterations to properly develop a model - therefore, it's
  likely that the slopes of certain predictors on the left are going to
  increase in magnitude at a greater rate than of those on the right.
\end{enumerate}

\end{document}
